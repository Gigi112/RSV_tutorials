% Options for packages loaded elsewhere
\PassOptionsToPackage{unicode}{hyperref}
\PassOptionsToPackage{hyphens}{url}
%
\documentclass[
]{article}
\usepackage{amsmath,amssymb}
\usepackage{lmodern}
\usepackage{iftex}
\ifPDFTeX
  \usepackage[T1]{fontenc}
  \usepackage[utf8]{inputenc}
  \usepackage{textcomp} % provide euro and other symbols
\else % if luatex or xetex
  \usepackage{unicode-math}
  \defaultfontfeatures{Scale=MatchLowercase}
  \defaultfontfeatures[\rmfamily]{Ligatures=TeX,Scale=1}
\fi
% Use upquote if available, for straight quotes in verbatim environments
\IfFileExists{upquote.sty}{\usepackage{upquote}}{}
\IfFileExists{microtype.sty}{% use microtype if available
  \usepackage[]{microtype}
  \UseMicrotypeSet[protrusion]{basicmath} % disable protrusion for tt fonts
}{}
\makeatletter
\@ifundefined{KOMAClassName}{% if non-KOMA class
  \IfFileExists{parskip.sty}{%
    \usepackage{parskip}
  }{% else
    \setlength{\parindent}{0pt}
    \setlength{\parskip}{6pt plus 2pt minus 1pt}}
}{% if KOMA class
  \KOMAoptions{parskip=half}}
\makeatother
\usepackage{xcolor}
\usepackage[margin=1in]{geometry}
\usepackage{graphicx}
\makeatletter
\def\maxwidth{\ifdim\Gin@nat@width>\linewidth\linewidth\else\Gin@nat@width\fi}
\def\maxheight{\ifdim\Gin@nat@height>\textheight\textheight\else\Gin@nat@height\fi}
\makeatother
% Scale images if necessary, so that they will not overflow the page
% margins by default, and it is still possible to overwrite the defaults
% using explicit options in \includegraphics[width, height, ...]{}
\setkeys{Gin}{width=\maxwidth,height=\maxheight,keepaspectratio}
% Set default figure placement to htbp
\makeatletter
\def\fps@figure{htbp}
\makeatother
\setlength{\emergencystretch}{3em} % prevent overfull lines
\providecommand{\tightlist}{%
  \setlength{\itemsep}{0pt}\setlength{\parskip}{0pt}}
\setcounter{secnumdepth}{-\maxdimen} % remove section numbering
\ifLuaTeX
  \usepackage{selnolig}  % disable illegal ligatures
\fi
\IfFileExists{bookmark.sty}{\usepackage{bookmark}}{\usepackage{hyperref}}
\IfFileExists{xurl.sty}{\usepackage{xurl}}{} % add URL line breaks if available
\urlstyle{same} % disable monospaced font for URLs
\hypersetup{
  pdftitle={Scale hospitalizations},
  pdfauthor={ZHE ZHENG},
  hidelinks,
  pdfcreator={LaTeX via pandoc}}

\title{Scale hospitalizations}
\author{ZHE ZHENG}
\date{2022-11-16}

\begin{document}
\maketitle

\hypertarget{estimate-the-true-burden-of-hospitalizations-across-risk-and-age-groups-in-older-adults}{%
\subsection{Estimate the true burden of hospitalizations across risk and
age groups in older
adults}\label{estimate-the-true-burden-of-hospitalizations-across-risk-and-age-groups-in-older-adults}}

\hypertarget{background}{%
\subparagraph{Background}\label{background}}

Risk groups will be defined by The Elixhauser Comorbidity Indices.
{(Question: how to stratify risk groups based on the indices?} For
example, scores are grouped into categories of less than 0, 0, 1--4, and
5 or higher. Chronic pulmonary disease is 3 points and Congestive heart
failure is 7 points) references:
\href{https://www.jstor.org/stable/40221931?seq=7\#metadata_info_tab_contents}{The
Elixhauser Comorbidity Index score and the probablity of inpatient
death}

\begin{itemize}
\tightlist
\item
  Low risk population: Index \textless= 0
\item
  Medium risk population: Index 1-4
\item
  High risk population: 5 or higher
\end{itemize}

\hypertarget{estimate-the-true-burden-of-rsv-hospitalizations-and-ed-visits}{%
\subparagraph{Estimate the ``true'' burden of RSV hospitalizations and
ED
visits}\label{estimate-the-true-burden-of-rsv-hospitalizations-and-ed-visits}}

Model structure:\\
\(Y_{ijk}\) \textasciitilde{} Negative Binomial (\(p_{ijk}\),\(r\))\\
\(p_{ijk}=r/(r+λ_{ijk})\)\\
where \(Y_{ijk}\) denotes the number of all-cause respiratory
hospitalizations at time (month) \(i\), in age group \(j\), and risk
group \(k\). \(i = 1,2,...,48\)\\
We will analyze hospitalizations in 11 age categories (\textless1, 1,
2-4, 5-9, 10-19, 20-59, 60-64, 65-70, 70-74, 75-79, 80+ years).
\(j = 1,2,...,12\)\\
\strut \\
We define the expected value as a function of covariates and random
effects such that:

\[\lambda_{ijk}=\beta_{0jk}+\alpha_{1g(i)}+\alpha_{2m(i)}+\beta_{1jk}RSV_{ik}+\beta_{2g(i)jk}Flu_{ik}\]
{Question 1: Should we use risk group specific RSV indicator
\(RSV_{ik}\)} (means RSV hospitalizations in high-risk or low-risk
children) or{ \(RSV_{i}\) } (means every risk group has the same RSV
timing) ?\\
\strut \\
The number of RSV hospitalizations/ED visits in each age group and risk
will be {\(\beta_{1jk}RSV_{ik}\)}

\hypertarget{estimate-death-rate-in-each-risk-group}{%
\subparagraph{Estimate death rate in each risk
group}\label{estimate-death-rate-in-each-risk-group}}

\begin{enumerate}
\def\labelenumi{(\arabic{enumi})}
\tightlist
\item
  Include DIED element in SAS code
\item
  Calculate the death rate in each risk and age group and assume that
  the death rate is the same between reported cases and the cases that
  did not reported
\end{enumerate}

\hypertarget{fitting-transmission-model-to-time-series-output}{%
\subparagraph{Fitting transmission model to time-series
output}\label{fitting-transmission-model-to-time-series-output}}

Parameters to estimate (in total 6?):

\begin{enumerate}
\def\labelenumi{(\arabic{enumi})}
\tightlist
\item
  Transmission parameter (related to transmission probability)
\item
  Amplitude of seasonality
\item
  Timing of seasonality
\item
  Duration of maternal immunity
\item
  Proportion of third LRIs that are hospitalized in older adults (if we
  assume their contact patterns are similar, we only need to estimate
  two parameters)
\end{enumerate}

\begin{itemize}
\tightlist
\item
  60-64 (low \(H_{60l}\), medium
  \(\frac{\beta_{60m}}{\beta_{60l}} H_{60l}\), and high risk
  \(\frac{\beta_{60h}}{\beta_{60l}} H_{60l}\))
\item
  65-69 (low \(\frac{\beta_{65l}}{\beta_{60l}}H_{60l}\), medium
  \(\frac{\beta_{65m}}{\beta_{60l}}H_{60l}\), and high risk
  \(\frac{\beta_{65h}}{\beta_{60l}}H_{60l}\))
\item
  70-74 (low \(\frac{\beta_{70l}}{\beta_{60l}}H_{60l}\), medium
  \(\frac{\beta_{70m}}{\beta_{60l}}H_{60l}\), and high risk
  \(\frac{\beta_{70h}}{\beta_{60l}}H_{60l}\))
\item
  75-79 (low \(\frac{\beta_{75l}}{\beta_{60l}}H_{60l}\), medium
  \(\frac{\beta_{75m}}{\beta_{60l}}H_{60l}\), and high risk
  \(\frac{\beta_{75h}}{\beta_{60l}}H_{60l}\))
\item
  80+ years (low \(H_{80l}\), medium
  \(\frac{\beta_{80m}}{\beta_{80l}}H_{80l}\), and high risk
  \(\frac{\beta_{80h}}{\beta_{80l}}H_{80l}\))\\
\end{itemize}

\hypertarget{vaccine-efficacy}{%
\subparagraph{Vaccine efficacy}\label{vaccine-efficacy}}

References:

\begin{enumerate}
\def\labelenumi{(\arabic{enumi})}
\tightlist
\item
  \href{https://www.nejm.org/doi/full/10.1056/NEJMoa2116154}{Efficacy on
  infectiousness,preF vaccine}
\item
  \href{https://academic.oup.com/jid/article/226/3/396/6064820}{Efficacy
  on infectiousness, vector vaccine}
\item
  \href{https://www.pfizer.com/news/press-release/press-release-detail/pfizer-announces-positive-top-line-data-phase-3-trial-older}{Efficacy
  on symptomatic infection, Pfizer}
\item
  \href{https://www.gsk.com/en-gb/media/press-releases/gsk-s-older-adult-respiratory-syncytial-virus-rsv-vaccine-candidate/}{Efficacy
  on symptomatic infection, GSK}
\item
  \href{https://www.jnj.com/janssen-announces-phase-2b-data-demonstrating-its-investigational-rsv-adult-vaccine-provided-80-protection-against-lower-respiratory-infections-in-older-adults}{Efficacy
  on symptomatic infection, Janssen}
\end{enumerate}

vaccine effects:

\begin{enumerate}
\def\labelenumi{(\arabic{enumi})}
\tightlist
\item
  Reduce the risk of infection? \((1-VE_{s})\sigma_3\lambda\)
\end{enumerate}

~where \(VE_{s}\) is the vaccine efficacy against infection,
\(\sigma_3\) is the relative susceptibility during subsequent
infections, and \(\lambda\) is the force of transmission.

\begin{enumerate}
\def\labelenumi{(\arabic{enumi})}
\setcounter{enumi}{1}
\tightlist
\item
  Reduce the infectiousness of vaccinees? \((1-VE_{i})I_v\)
\end{enumerate}

~where \(VE_{i}\) is the vaccine efficacy to reduce infectiousness and
\(I_v\) is the number of vaccinated individuals who are infected by
wild-type RSV

\end{document}
